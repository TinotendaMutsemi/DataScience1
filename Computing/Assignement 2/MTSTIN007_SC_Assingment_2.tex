% Options for packages loaded elsewhere
\PassOptionsToPackage{unicode}{hyperref}
\PassOptionsToPackage{hyphens}{url}
%
\documentclass[
]{article}
\usepackage{amsmath,amssymb}
\usepackage{iftex}
\ifPDFTeX
  \usepackage[T1]{fontenc}
  \usepackage[utf8]{inputenc}
  \usepackage{textcomp} % provide euro and other symbols
\else % if luatex or xetex
  \usepackage{unicode-math} % this also loads fontspec
  \defaultfontfeatures{Scale=MatchLowercase}
  \defaultfontfeatures[\rmfamily]{Ligatures=TeX,Scale=1}
\fi
\usepackage{lmodern}
\ifPDFTeX\else
  % xetex/luatex font selection
\fi
% Use upquote if available, for straight quotes in verbatim environments
\IfFileExists{upquote.sty}{\usepackage{upquote}}{}
\IfFileExists{microtype.sty}{% use microtype if available
  \usepackage[]{microtype}
  \UseMicrotypeSet[protrusion]{basicmath} % disable protrusion for tt fonts
}{}
\makeatletter
\@ifundefined{KOMAClassName}{% if non-KOMA class
  \IfFileExists{parskip.sty}{%
    \usepackage{parskip}
  }{% else
    \setlength{\parindent}{0pt}
    \setlength{\parskip}{6pt plus 2pt minus 1pt}}
}{% if KOMA class
  \KOMAoptions{parskip=half}}
\makeatother
\usepackage{xcolor}
\usepackage[margin=1in]{geometry}
\usepackage{color}
\usepackage{fancyvrb}
\newcommand{\VerbBar}{|}
\newcommand{\VERB}{\Verb[commandchars=\\\{\}]}
\DefineVerbatimEnvironment{Highlighting}{Verbatim}{commandchars=\\\{\}}
% Add ',fontsize=\small' for more characters per line
\usepackage{framed}
\definecolor{shadecolor}{RGB}{248,248,248}
\newenvironment{Shaded}{\begin{snugshade}}{\end{snugshade}}
\newcommand{\AlertTok}[1]{\textcolor[rgb]{0.94,0.16,0.16}{#1}}
\newcommand{\AnnotationTok}[1]{\textcolor[rgb]{0.56,0.35,0.01}{\textbf{\textit{#1}}}}
\newcommand{\AttributeTok}[1]{\textcolor[rgb]{0.13,0.29,0.53}{#1}}
\newcommand{\BaseNTok}[1]{\textcolor[rgb]{0.00,0.00,0.81}{#1}}
\newcommand{\BuiltInTok}[1]{#1}
\newcommand{\CharTok}[1]{\textcolor[rgb]{0.31,0.60,0.02}{#1}}
\newcommand{\CommentTok}[1]{\textcolor[rgb]{0.56,0.35,0.01}{\textit{#1}}}
\newcommand{\CommentVarTok}[1]{\textcolor[rgb]{0.56,0.35,0.01}{\textbf{\textit{#1}}}}
\newcommand{\ConstantTok}[1]{\textcolor[rgb]{0.56,0.35,0.01}{#1}}
\newcommand{\ControlFlowTok}[1]{\textcolor[rgb]{0.13,0.29,0.53}{\textbf{#1}}}
\newcommand{\DataTypeTok}[1]{\textcolor[rgb]{0.13,0.29,0.53}{#1}}
\newcommand{\DecValTok}[1]{\textcolor[rgb]{0.00,0.00,0.81}{#1}}
\newcommand{\DocumentationTok}[1]{\textcolor[rgb]{0.56,0.35,0.01}{\textbf{\textit{#1}}}}
\newcommand{\ErrorTok}[1]{\textcolor[rgb]{0.64,0.00,0.00}{\textbf{#1}}}
\newcommand{\ExtensionTok}[1]{#1}
\newcommand{\FloatTok}[1]{\textcolor[rgb]{0.00,0.00,0.81}{#1}}
\newcommand{\FunctionTok}[1]{\textcolor[rgb]{0.13,0.29,0.53}{\textbf{#1}}}
\newcommand{\ImportTok}[1]{#1}
\newcommand{\InformationTok}[1]{\textcolor[rgb]{0.56,0.35,0.01}{\textbf{\textit{#1}}}}
\newcommand{\KeywordTok}[1]{\textcolor[rgb]{0.13,0.29,0.53}{\textbf{#1}}}
\newcommand{\NormalTok}[1]{#1}
\newcommand{\OperatorTok}[1]{\textcolor[rgb]{0.81,0.36,0.00}{\textbf{#1}}}
\newcommand{\OtherTok}[1]{\textcolor[rgb]{0.56,0.35,0.01}{#1}}
\newcommand{\PreprocessorTok}[1]{\textcolor[rgb]{0.56,0.35,0.01}{\textit{#1}}}
\newcommand{\RegionMarkerTok}[1]{#1}
\newcommand{\SpecialCharTok}[1]{\textcolor[rgb]{0.81,0.36,0.00}{\textbf{#1}}}
\newcommand{\SpecialStringTok}[1]{\textcolor[rgb]{0.31,0.60,0.02}{#1}}
\newcommand{\StringTok}[1]{\textcolor[rgb]{0.31,0.60,0.02}{#1}}
\newcommand{\VariableTok}[1]{\textcolor[rgb]{0.00,0.00,0.00}{#1}}
\newcommand{\VerbatimStringTok}[1]{\textcolor[rgb]{0.31,0.60,0.02}{#1}}
\newcommand{\WarningTok}[1]{\textcolor[rgb]{0.56,0.35,0.01}{\textbf{\textit{#1}}}}
\usepackage{longtable,booktabs,array}
\usepackage{calc} % for calculating minipage widths
% Correct order of tables after \paragraph or \subparagraph
\usepackage{etoolbox}
\makeatletter
\patchcmd\longtable{\par}{\if@noskipsec\mbox{}\fi\par}{}{}
\makeatother
% Allow footnotes in longtable head/foot
\IfFileExists{footnotehyper.sty}{\usepackage{footnotehyper}}{\usepackage{footnote}}
\makesavenoteenv{longtable}
\usepackage{graphicx}
\makeatletter
\def\maxwidth{\ifdim\Gin@nat@width>\linewidth\linewidth\else\Gin@nat@width\fi}
\def\maxheight{\ifdim\Gin@nat@height>\textheight\textheight\else\Gin@nat@height\fi}
\makeatother
% Scale images if necessary, so that they will not overflow the page
% margins by default, and it is still possible to overwrite the defaults
% using explicit options in \includegraphics[width, height, ...]{}
\setkeys{Gin}{width=\maxwidth,height=\maxheight,keepaspectratio}
% Set default figure placement to htbp
\makeatletter
\def\fps@figure{htbp}
\makeatother
\setlength{\emergencystretch}{3em} % prevent overfull lines
\providecommand{\tightlist}{%
  \setlength{\itemsep}{0pt}\setlength{\parskip}{0pt}}
\setcounter{secnumdepth}{-\maxdimen} % remove section numbering
\ifLuaTeX
  \usepackage{selnolig}  % disable illegal ligatures
\fi
\IfFileExists{bookmark.sty}{\usepackage{bookmark}}{\usepackage{hyperref}}
\IfFileExists{xurl.sty}{\usepackage{xurl}}{} % add URL line breaks if available
\urlstyle{same}
\hypersetup{
  pdftitle={Statistical Computing},
  pdfauthor={Tinotenda Mutsemi (MTSTIN007)},
  hidelinks,
  pdfcreator={LaTeX via pandoc}}

\title{Statistical Computing}
\usepackage{etoolbox}
\makeatletter
\providecommand{\subtitle}[1]{% add subtitle to \maketitle
  \apptocmd{\@title}{\par {\large #1 \par}}{}{}
}
\makeatother
\subtitle{Assignment 2}
\author{Tinotenda Mutsemi (MTSTIN007)}
\date{2024-03-05}

\begin{document}
\maketitle

\hypertarget{question-1}{%
\subsection{Question 1}\label{question-1}}

First we read in the 4 datasets. The datasets are:

\begin{itemize}
\item
  lex.csv
\item
  gdp\_pcap.csv
\item
  Data Geographies - v2 - by Gapminder.xlsx
\item
  GM-Population - Dataset - v6.xlsx
\end{itemize}

Below is the code.

\begin{Shaded}
\begin{Highlighting}[]
\CommentTok{\#read data}
\NormalTok{lex\_raw }\OtherTok{\textless{}{-}} \FunctionTok{read.csv}\NormalTok{(}\StringTok{"lex.csv"}\NormalTok{)}
\NormalTok{gdp\_pcap\_raw }\OtherTok{\textless{}{-}} \FunctionTok{read.csv}\NormalTok{(}\StringTok{"gdp\_pcap.csv"}\NormalTok{)}
\NormalTok{region\_raw }\OtherTok{\textless{}{-}} \FunctionTok{read.xlsx}\NormalTok{(}\StringTok{"Data Geographies {-} v2 {-} by Gapminder.xlsx"}\NormalTok{, }\AttributeTok{sheetIndex =} \DecValTok{2}\NormalTok{)}
\NormalTok{population\_raw }\OtherTok{\textless{}{-}} \FunctionTok{read.xlsx}\NormalTok{(}\StringTok{"GM{-}Population {-} Dataset {-} v6.xlsx"}\NormalTok{, }\AttributeTok{sheetIndex =} \DecValTok{4}\NormalTok{)}
\end{Highlighting}
\end{Shaded}

\hypertarget{question-1a}{%
\subsubsection{Question 1a}\label{question-1a}}

Now we will clean data and select the 2019 columns. Before we plot the
data, we will merge the datasets and clean the dataframes. Using full
join to get all the data from the columns. This creates tidy data for
ggplot.

We will plot the life expectancy vs GDP per capita for each country in
2019. The size of the points will be determined by the population of the
country. The colour of the points will be determined by the region of
the country.

\begin{figure}
\centering
\includegraphics[width=11.8cm,height=\textheight]{lex_gdp_region_popu_2019.png}
\caption{2019 Life Expectancy vs GDP per Capita}
\end{figure}

Below is the code for the detailed procedure to generate figure 1.

\begin{Shaded}
\begin{Highlighting}[]
\CommentTok{\#data cleaning and selecting required data}
\NormalTok{lex\_2019 }\OtherTok{\textless{}{-}} \FunctionTok{select}\NormalTok{(lex\_raw, country, X2019)}

\NormalTok{gdp\_pcap\_2019 }\OtherTok{\textless{}{-}} \FunctionTok{select}\NormalTok{(gdp\_pcap\_raw, country, X2019)}

\CommentTok{\#region data}
\CommentTok{\#rename region col names}
\NormalTok{lookup }\OtherTok{\textless{}{-}} \FunctionTok{c}\NormalTok{(}\AttributeTok{country =} \StringTok{"name"}\NormalTok{,}
            \AttributeTok{region =} \StringTok{"four\_regions"}\NormalTok{)}
\NormalTok{region }\OtherTok{\textless{}{-}} \FunctionTok{select}\NormalTok{(region\_raw, name, four\_regions) }\SpecialCharTok{|\textgreater{}} \FunctionTok{rename}\NormalTok{(}\FunctionTok{all\_of}\NormalTok{(lookup))}


\CommentTok{\#population data}
\CommentTok{\#deselect col}
\NormalTok{population\_raw2 }\OtherTok{\textless{}{-}} \FunctionTok{select}\NormalTok{(population\_raw, }\SpecialCharTok{{-}}\NormalTok{geo)}
\CommentTok{\#rename cols}
\NormalTok{lookup }\OtherTok{\textless{}{-}} \FunctionTok{c}\NormalTok{(}\AttributeTok{country =} \StringTok{"name"}\NormalTok{,}
            \AttributeTok{year =} \StringTok{"time"}\NormalTok{,}
            \AttributeTok{population =} \StringTok{"Population"}\NormalTok{)}

\NormalTok{population\_raw2 }\OtherTok{\textless{}{-}} \FunctionTok{rename}\NormalTok{(population\_raw2, }\FunctionTok{all\_of}\NormalTok{(lookup))}

\CommentTok{\#select 2019 population}
\NormalTok{population\_2019 }\OtherTok{\textless{}{-}} \FunctionTok{filter}\NormalTok{(population\_raw2, year }\SpecialCharTok{==} \DecValTok{2019}\NormalTok{)}
\end{Highlighting}
\end{Shaded}

\begin{Shaded}
\begin{Highlighting}[]
\CommentTok{\#merge and clean data frames}
\NormalTok{lex\_gdp\_2019 }\OtherTok{\textless{}{-}} \FunctionTok{full\_join}\NormalTok{(lex\_2019, gdp\_pcap\_2019, }\AttributeTok{by =} \FunctionTok{join\_by}\NormalTok{(country))}
\NormalTok{lex\_gdp\_region\_2019 }\OtherTok{\textless{}{-}} \FunctionTok{full\_join}\NormalTok{(lex\_gdp\_2019, region, }\AttributeTok{by =} \FunctionTok{join\_by}\NormalTok{(country))}

\CommentTok{\#rename cols of new df}
\NormalTok{lookup }\OtherTok{\textless{}{-}} \FunctionTok{c}\NormalTok{(}\AttributeTok{lex =} \StringTok{"X2019.x"}\NormalTok{, }
            \AttributeTok{gdp\_pcap =} \StringTok{"X2019.y"}\NormalTok{)}
\NormalTok{lex\_gdp\_region\_2019 }\OtherTok{\textless{}{-}} \FunctionTok{rename}\NormalTok{(lex\_gdp\_region\_2019, }\FunctionTok{all\_of}\NormalTok{(lookup))}

\CommentTok{\#make gdp\_pcap numeric}
\NormalTok{lex\_gdp\_region\_2019}\SpecialCharTok{$}\NormalTok{gdp\_pcap }\OtherTok{\textless{}{-}} \FunctionTok{as.numeric}\NormalTok{(lex\_gdp\_region\_2019}\SpecialCharTok{$}\NormalTok{gdp\_pcap)}

\NormalTok{lex\_gdp\_region\_popu\_2019 }\OtherTok{\textless{}{-}} \FunctionTok{full\_join}\NormalTok{(lex\_gdp\_region\_2019, population\_2019, }\AttributeTok{by =} \FunctionTok{join\_by}\NormalTok{(country))}
\end{Highlighting}
\end{Shaded}

\begin{Shaded}
\begin{Highlighting}[]
\NormalTok{plot\_lex\_gdp\_region\_popu\_2019 }\OtherTok{\textless{}{-}} \FunctionTok{na.omit}\NormalTok{(lex\_gdp\_region\_popu\_2019)}
\CommentTok{\#scatter plot with ggplot}
\NormalTok{plot\_lex\_gdp\_region\_popu\_2019 }\OtherTok{\textless{}{-}} \FunctionTok{ggplot}\NormalTok{(}\AttributeTok{data =}\NormalTok{ plot\_lex\_gdp\_region\_popu\_2019, }\AttributeTok{mapping =} \FunctionTok{aes}\NormalTok{(}\AttributeTok{x =}\NormalTok{ gdp\_pcap, }\AttributeTok{y =}\NormalTok{ lex,}
             \AttributeTok{colour =}\NormalTok{ region,}
             \AttributeTok{size =}\NormalTok{ population,}
\NormalTok{             )) }\SpecialCharTok{+}
  \FunctionTok{geom\_point}\NormalTok{() }\SpecialCharTok{+}
  \FunctionTok{geom\_text}\NormalTok{(}\FunctionTok{aes}\NormalTok{(}\AttributeTok{label =}\NormalTok{ country), }\AttributeTok{check\_overlap =} \ConstantTok{TRUE}\NormalTok{, }\AttributeTok{vjust =} \DecValTok{1}\NormalTok{, }\AttributeTok{hjust =} \DecValTok{1}\NormalTok{, }\AttributeTok{col =} \StringTok{"black"}\NormalTok{) }\SpecialCharTok{+}
  \FunctionTok{scale\_x\_log10}\NormalTok{() }\SpecialCharTok{+}
  \FunctionTok{scale\_y\_log10}\NormalTok{() }\SpecialCharTok{+}
  \FunctionTok{labs}\NormalTok{(}
       \AttributeTok{x =} \StringTok{"GDP per capita"}\NormalTok{,}
       \AttributeTok{y =} \StringTok{"Life expectancy"}\NormalTok{)}
  

  \CommentTok{\# title = "GDP per capita vs Life expectancy 2019",}
\CommentTok{\#save plot}
\FunctionTok{ggsave}\NormalTok{(}\StringTok{"lex\_gdp\_region\_popu\_2019.png"}\NormalTok{, }\AttributeTok{plot =}\NormalTok{ plot\_lex\_gdp\_region\_popu\_2019)}
\end{Highlighting}
\end{Shaded}

\hypertarget{question-1b}{%
\subsubsection{Question 1b}\label{question-1b}}

We will now calculate the average life expectancy for each region and
the number of countries in each region.

\begin{longtable}[]{@{}lll@{}}
\caption{Table of average life expectancy and the number of countries in
each region.}\tabularnewline
\toprule\noalign{}
Region & Ave Life Exp & Num of Countries \\
\midrule\noalign{}
\endfirsthead
\toprule\noalign{}
Region & Ave Life Exp & Num of Countries \\
\midrule\noalign{}
\endhead
\bottomrule\noalign{}
\endlastfoot
Europe & 79.1 & 49 \\
Americas & 75.2 & 35 \\
Asia & 73.0 & 59 \\
Africa & 65.9 & 54 \\
\end{longtable}

The code below generated the Table 1.

\begin{Shaded}
\begin{Highlighting}[]
\CommentTok{\#Qtn b}
\NormalTok{region\_avg\_lex }\OtherTok{\textless{}{-}}\NormalTok{ lex\_gdp\_region\_popu\_2019 }\SpecialCharTok{|\textgreater{}} \FunctionTok{group\_by}\NormalTok{(region) }\SpecialCharTok{|\textgreater{}} \FunctionTok{summarise}\NormalTok{(}\AttributeTok{avg\_region\_lex =} \FunctionTok{mean}\NormalTok{(lex, }\AttributeTok{na.rm =} \ConstantTok{TRUE}\NormalTok{), }\AttributeTok{countries\_in\_region =} \FunctionTok{n}\NormalTok{())}

\CommentTok{\#remove na region}
\NormalTok{region\_avg\_lex }\OtherTok{\textless{}{-}} \FunctionTok{na.omit}\NormalTok{(region\_avg\_lex)}

\CommentTok{\#sort by avg\_region\_lex descending}
\NormalTok{region\_avg\_lex }\OtherTok{\textless{}{-}}\NormalTok{ region\_avg\_lex }\SpecialCharTok{|\textgreater{}} \FunctionTok{arrange}\NormalTok{(}\FunctionTok{desc}\NormalTok{(avg\_region\_lex))}
\CommentTok{\# region\_avg\_lex}
\end{Highlighting}
\end{Shaded}

\hypertarget{question-1c}{%
\subsubsection{Question 1c}\label{question-1c}}

We will now calculate the average life expectancy for each region and
year. We will then plot the average life expectancy for each region over
time.

\begin{figure}
\centering
\includegraphics[width=11.6cm,height=\textheight]{region_avg_lex.png}
\caption{Average Life Expectancy for each region.}
\end{figure}

The code below generated Figure 2.

\begin{Shaded}
\begin{Highlighting}[]
\CommentTok{\#Qtn c}

\CommentTok{\#melt lex\_raw}
\NormalTok{lex\_melt }\OtherTok{\textless{}{-}} \FunctionTok{melt}\NormalTok{(lex\_raw, }\AttributeTok{id.vars =} \StringTok{"country"}\NormalTok{, }\AttributeTok{value.name =} \StringTok{"lex"}\NormalTok{, }\AttributeTok{variable.name =} \StringTok{"year"}\NormalTok{)}



\CommentTok{\#merge lex\_melt with regions}
\NormalTok{lex\_region }\OtherTok{\textless{}{-}} \FunctionTok{full\_join}\NormalTok{(lex\_melt, region, }\AttributeTok{by =} \StringTok{"country"}\NormalTok{)}

\NormalTok{region\_avg\_lex }\OtherTok{\textless{}{-}}\NormalTok{ lex\_region }\SpecialCharTok{|\textgreater{}} \FunctionTok{group\_by}\NormalTok{(region, year) }\SpecialCharTok{|\textgreater{}} \FunctionTok{summarise}\NormalTok{(}\AttributeTok{avg\_lex =} \FunctionTok{mean}\NormalTok{(lex, }\AttributeTok{na.rm =} \ConstantTok{TRUE}\NormalTok{))}

\CommentTok{\#remove leading X from year}
\NormalTok{region\_avg\_lex}\SpecialCharTok{$}\NormalTok{year }\OtherTok{\textless{}{-}} \FunctionTok{gsub}\NormalTok{(}\StringTok{"X"}\NormalTok{, }\StringTok{""}\NormalTok{, region\_avg\_lex}\SpecialCharTok{$}\NormalTok{year)}
\CommentTok{\#make year numeric}
\NormalTok{region\_avg\_lex}\SpecialCharTok{$}\NormalTok{year }\OtherTok{\textless{}{-}} \FunctionTok{as.numeric}\NormalTok{(region\_avg\_lex}\SpecialCharTok{$}\NormalTok{year)}
\end{Highlighting}
\end{Shaded}

\begin{Shaded}
\begin{Highlighting}[]
\NormalTok{plot\_region\_avg\_lex }\OtherTok{=} \FunctionTok{na.omit}\NormalTok{(region\_avg\_lex)}
\NormalTok{plot\_region\_avg\_lex }\OtherTok{\textless{}{-}} \FunctionTok{ggplot}\NormalTok{(}\AttributeTok{data =}\NormalTok{ plot\_region\_avg\_lex, }\AttributeTok{mapping =} \FunctionTok{aes}\NormalTok{(}\AttributeTok{x =}\NormalTok{ year, }\AttributeTok{y =}\NormalTok{ avg\_lex, }\AttributeTok{group =}\NormalTok{ region, }\AttributeTok{col =}\NormalTok{ region)) }\SpecialCharTok{+}
  \FunctionTok{geom\_line}\NormalTok{() }\SpecialCharTok{+}
  \FunctionTok{labs}\NormalTok{(}
       \AttributeTok{x =} \StringTok{"Year"}\NormalTok{,}
       \AttributeTok{y =} \StringTok{"Average life expectancy"}\NormalTok{) }\SpecialCharTok{+}
  \FunctionTok{scale\_x\_continuous}\NormalTok{(}\AttributeTok{breaks =} \FunctionTok{seq}\NormalTok{(}\DecValTok{1800}\NormalTok{, }\DecValTok{2160}\NormalTok{,}\DecValTok{20}\NormalTok{)) }\SpecialCharTok{+}
  \FunctionTok{scale\_y\_continuous}\NormalTok{(}\AttributeTok{breaks =} \FunctionTok{seq}\NormalTok{(}\DecValTok{0}\NormalTok{, }\DecValTok{90}\NormalTok{, }\DecValTok{10}\NormalTok{)) }\SpecialCharTok{+}
  \FunctionTok{theme}\NormalTok{(}\AttributeTok{axis.text.x =} \FunctionTok{element\_text}\NormalTok{(}\AttributeTok{angle =} \DecValTok{45}\NormalTok{, }\AttributeTok{vjust =} \FloatTok{0.5}\NormalTok{, }\AttributeTok{hjust =} \DecValTok{1}\NormalTok{))}
  
\CommentTok{\# title = "Region average life expectancy over time",}

\CommentTok{\#save plot}
\FunctionTok{ggsave}\NormalTok{(}\StringTok{"region\_avg\_lex.png"}\NormalTok{, }\AttributeTok{plot =}\NormalTok{ plot\_region\_avg\_lex)}
\end{Highlighting}
\end{Shaded}

\hypertarget{question-1d}{%
\subsubsection{Question 1d}\label{question-1d}}

We will now plot the average life expectancy for each region in 2019. We
will use a bar chart to plot the data.

\begin{figure}
\centering
\includegraphics[width=4.03125in,height=\textheight]{region_avg_lex_2019.png}
\caption{2019 Average Life Expectancy for each region}
\end{figure}

\begin{Shaded}
\begin{Highlighting}[]
\CommentTok{\#Qtn d}
\CommentTok{\#select 2019 data}
\NormalTok{region\_avg\_lex\_2019 }\OtherTok{\textless{}{-}} \FunctionTok{filter}\NormalTok{(region\_avg\_lex, year }\SpecialCharTok{==} \DecValTok{2019}\NormalTok{)}
\CommentTok{\#drop na region}
\NormalTok{region\_avg\_lex\_2019 }\OtherTok{\textless{}{-}} \FunctionTok{na.omit}\NormalTok{(region\_avg\_lex\_2019)}

\CommentTok{\#plot bar chart}
\NormalTok{region\_avg\_lex\_2019 }\OtherTok{\textless{}{-}} \FunctionTok{ggplot}\NormalTok{(}\AttributeTok{data =}\NormalTok{ region\_avg\_lex\_2019, }\AttributeTok{mapping =} \FunctionTok{aes}\NormalTok{(}\AttributeTok{x =}\NormalTok{ region, }\AttributeTok{y =}\NormalTok{ avg\_lex)) }\SpecialCharTok{+}
  \FunctionTok{geom\_bar}\NormalTok{(}\AttributeTok{stat =} \StringTok{"identity"}\NormalTok{) }\SpecialCharTok{+}
  \FunctionTok{labs}\NormalTok{(}
       \AttributeTok{x =} \StringTok{"Region"}\NormalTok{,}
       \AttributeTok{y =} \StringTok{"Average life expectancy"}\NormalTok{) }\SpecialCharTok{+}
  \CommentTok{\# theme(axis.text.x = element\_text(angle = 45, vjust = 0.5, hjust = 1)) +}
  \FunctionTok{scale\_y\_continuous}\NormalTok{(}\AttributeTok{breaks =} \FunctionTok{seq}\NormalTok{(}\DecValTok{0}\NormalTok{, }\DecValTok{90}\NormalTok{, }\DecValTok{10}\NormalTok{)) }\SpecialCharTok{+}
  \FunctionTok{geom\_text}\NormalTok{(}\FunctionTok{aes}\NormalTok{(}\AttributeTok{label =} \FunctionTok{round}\NormalTok{(avg\_lex, }\DecValTok{0}\NormalTok{)), }\AttributeTok{vjust =} \SpecialCharTok{{-}}\FloatTok{0.5}\NormalTok{, }\AttributeTok{hjust =} \DecValTok{1}\NormalTok{, }\AttributeTok{col =} \StringTok{"black"}\NormalTok{)}

\CommentTok{\# title = "Region average life expectancy 2019",}

\CommentTok{\#save plot}
\FunctionTok{ggsave}\NormalTok{(}\StringTok{"region\_avg\_lex\_2019.png"}\NormalTok{, }\AttributeTok{plot =}\NormalTok{ region\_avg\_lex\_2019)}
\end{Highlighting}
\end{Shaded}

\hypertarget{question-1e}{%
\subsubsection{Question 1e}\label{question-1e}}

We will now calculate the number of countries that have a two-word name.
The number of countries that have a two-word name is 24

The code below uses strsplit and sapply to calculate the number of
countries that have a two-word name.

\begin{Shaded}
\begin{Highlighting}[]
\CommentTok{\#Qtn e}
\NormalTok{country\_split }\OtherTok{\textless{}{-}} \FunctionTok{strsplit}\NormalTok{(lex\_raw}\SpecialCharTok{$}\NormalTok{country, }\StringTok{" "}\NormalTok{)}

\CommentTok{\#get count two word countries}
\NormalTok{two\_word\_countries }\OtherTok{\textless{}{-}} \FunctionTok{sum}\NormalTok{(}\FunctionTok{sapply}\NormalTok{(country\_split, length) }\SpecialCharTok{==} \DecValTok{2}\NormalTok{)}
\CommentTok{\# two\_word\_countries}
\end{Highlighting}
\end{Shaded}


\end{document}
