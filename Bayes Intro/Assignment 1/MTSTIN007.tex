% Options for packages loaded elsewhere
\PassOptionsToPackage{unicode}{hyperref}
\PassOptionsToPackage{hyphens}{url}
%
\documentclass[
]{article}
\usepackage{amsmath,amssymb}
\usepackage{iftex}
\ifPDFTeX
  \usepackage[T1]{fontenc}
  \usepackage[utf8]{inputenc}
  \usepackage{textcomp} % provide euro and other symbols
\else % if luatex or xetex
  \usepackage{unicode-math} % this also loads fontspec
  \defaultfontfeatures{Scale=MatchLowercase}
  \defaultfontfeatures[\rmfamily]{Ligatures=TeX,Scale=1}
\fi
\usepackage{lmodern}
\ifPDFTeX\else
  % xetex/luatex font selection
\fi
% Use upquote if available, for straight quotes in verbatim environments
\IfFileExists{upquote.sty}{\usepackage{upquote}}{}
\IfFileExists{microtype.sty}{% use microtype if available
  \usepackage[]{microtype}
  \UseMicrotypeSet[protrusion]{basicmath} % disable protrusion for tt fonts
}{}
\makeatletter
\@ifundefined{KOMAClassName}{% if non-KOMA class
  \IfFileExists{parskip.sty}{%
    \usepackage{parskip}
  }{% else
    \setlength{\parindent}{0pt}
    \setlength{\parskip}{6pt plus 2pt minus 1pt}}
}{% if KOMA class
  \KOMAoptions{parskip=half}}
\makeatother
\usepackage{xcolor}
\usepackage[margin=1in]{geometry}
\usepackage{color}
\usepackage{fancyvrb}
\newcommand{\VerbBar}{|}
\newcommand{\VERB}{\Verb[commandchars=\\\{\}]}
\DefineVerbatimEnvironment{Highlighting}{Verbatim}{commandchars=\\\{\}}
% Add ',fontsize=\small' for more characters per line
\usepackage{framed}
\definecolor{shadecolor}{RGB}{248,248,248}
\newenvironment{Shaded}{\begin{snugshade}}{\end{snugshade}}
\newcommand{\AlertTok}[1]{\textcolor[rgb]{0.94,0.16,0.16}{#1}}
\newcommand{\AnnotationTok}[1]{\textcolor[rgb]{0.56,0.35,0.01}{\textbf{\textit{#1}}}}
\newcommand{\AttributeTok}[1]{\textcolor[rgb]{0.13,0.29,0.53}{#1}}
\newcommand{\BaseNTok}[1]{\textcolor[rgb]{0.00,0.00,0.81}{#1}}
\newcommand{\BuiltInTok}[1]{#1}
\newcommand{\CharTok}[1]{\textcolor[rgb]{0.31,0.60,0.02}{#1}}
\newcommand{\CommentTok}[1]{\textcolor[rgb]{0.56,0.35,0.01}{\textit{#1}}}
\newcommand{\CommentVarTok}[1]{\textcolor[rgb]{0.56,0.35,0.01}{\textbf{\textit{#1}}}}
\newcommand{\ConstantTok}[1]{\textcolor[rgb]{0.56,0.35,0.01}{#1}}
\newcommand{\ControlFlowTok}[1]{\textcolor[rgb]{0.13,0.29,0.53}{\textbf{#1}}}
\newcommand{\DataTypeTok}[1]{\textcolor[rgb]{0.13,0.29,0.53}{#1}}
\newcommand{\DecValTok}[1]{\textcolor[rgb]{0.00,0.00,0.81}{#1}}
\newcommand{\DocumentationTok}[1]{\textcolor[rgb]{0.56,0.35,0.01}{\textbf{\textit{#1}}}}
\newcommand{\ErrorTok}[1]{\textcolor[rgb]{0.64,0.00,0.00}{\textbf{#1}}}
\newcommand{\ExtensionTok}[1]{#1}
\newcommand{\FloatTok}[1]{\textcolor[rgb]{0.00,0.00,0.81}{#1}}
\newcommand{\FunctionTok}[1]{\textcolor[rgb]{0.13,0.29,0.53}{\textbf{#1}}}
\newcommand{\ImportTok}[1]{#1}
\newcommand{\InformationTok}[1]{\textcolor[rgb]{0.56,0.35,0.01}{\textbf{\textit{#1}}}}
\newcommand{\KeywordTok}[1]{\textcolor[rgb]{0.13,0.29,0.53}{\textbf{#1}}}
\newcommand{\NormalTok}[1]{#1}
\newcommand{\OperatorTok}[1]{\textcolor[rgb]{0.81,0.36,0.00}{\textbf{#1}}}
\newcommand{\OtherTok}[1]{\textcolor[rgb]{0.56,0.35,0.01}{#1}}
\newcommand{\PreprocessorTok}[1]{\textcolor[rgb]{0.56,0.35,0.01}{\textit{#1}}}
\newcommand{\RegionMarkerTok}[1]{#1}
\newcommand{\SpecialCharTok}[1]{\textcolor[rgb]{0.81,0.36,0.00}{\textbf{#1}}}
\newcommand{\SpecialStringTok}[1]{\textcolor[rgb]{0.31,0.60,0.02}{#1}}
\newcommand{\StringTok}[1]{\textcolor[rgb]{0.31,0.60,0.02}{#1}}
\newcommand{\VariableTok}[1]{\textcolor[rgb]{0.00,0.00,0.00}{#1}}
\newcommand{\VerbatimStringTok}[1]{\textcolor[rgb]{0.31,0.60,0.02}{#1}}
\newcommand{\WarningTok}[1]{\textcolor[rgb]{0.56,0.35,0.01}{\textbf{\textit{#1}}}}
\usepackage{longtable,booktabs,array}
\usepackage{calc} % for calculating minipage widths
% Correct order of tables after \paragraph or \subparagraph
\usepackage{etoolbox}
\makeatletter
\patchcmd\longtable{\par}{\if@noskipsec\mbox{}\fi\par}{}{}
\makeatother
% Allow footnotes in longtable head/foot
\IfFileExists{footnotehyper.sty}{\usepackage{footnotehyper}}{\usepackage{footnote}}
\makesavenoteenv{longtable}
\usepackage{graphicx}
\makeatletter
\def\maxwidth{\ifdim\Gin@nat@width>\linewidth\linewidth\else\Gin@nat@width\fi}
\def\maxheight{\ifdim\Gin@nat@height>\textheight\textheight\else\Gin@nat@height\fi}
\makeatother
% Scale images if necessary, so that they will not overflow the page
% margins by default, and it is still possible to overwrite the defaults
% using explicit options in \includegraphics[width, height, ...]{}
\setkeys{Gin}{width=\maxwidth,height=\maxheight,keepaspectratio}
% Set default figure placement to htbp
\makeatletter
\def\fps@figure{htbp}
\makeatother
\setlength{\emergencystretch}{3em} % prevent overfull lines
\providecommand{\tightlist}{%
  \setlength{\itemsep}{0pt}\setlength{\parskip}{0pt}}
\setcounter{secnumdepth}{-\maxdimen} % remove section numbering
\ifLuaTeX
  \usepackage{selnolig}  % disable illegal ligatures
\fi
\IfFileExists{bookmark.sty}{\usepackage{bookmark}}{\usepackage{hyperref}}
\IfFileExists{xurl.sty}{\usepackage{xurl}}{} % add URL line breaks if available
\urlstyle{same}
\hypersetup{
  pdftitle={Statistical Computing},
  pdfauthor={Tinotenda Mutsemi (MTSTIN007)},
  hidelinks,
  pdfcreator={LaTeX via pandoc}}

\title{Statistical Computing}
\usepackage{etoolbox}
\makeatletter
\providecommand{\subtitle}[1]{% add subtitle to \maketitle
  \apptocmd{\@title}{\par {\large #1 \par}}{}{}
}
\makeatother
\subtitle{Assignment 2}
\author{Tinotenda Mutsemi (MTSTIN007)}
\date{2024-03-05}

\begin{document}
\maketitle

\hypertarget{question-1}{%
\section{Question 1}\label{question-1}}

\hypertarget{question-1a}{%
\section{Question 1a}\label{question-1a}}

We show that conditional priors:

\[
[\beta|\sigma^2] \sim \mathcal{N}_{k+1}(\boldsymbol{\mu_\beta}, \sigma^2 (\boldsymbol{M + X^TX})^{-1})
\]

\[
[\sigma^2] \sim \mathcal{IG}(a + \frac{n}{2},b + \frac{A_2}{2})
\]

\hypertarget{question-1b}{%
\section{Question 1b}\label{question-1b}}

\begin{Shaded}
\begin{Highlighting}[]
\CommentTok{\#Using data from the preamble}


\CommentTok{\# Compute beta hat using the provided formula}
\NormalTok{beta.hat }\OtherTok{\textless{}{-}} \FunctionTok{solve}\NormalTok{(}\FunctionTok{t}\NormalTok{(X) }\SpecialCharTok{\%*\%}\NormalTok{ X) }\SpecialCharTok{\%*\%} \FunctionTok{t}\NormalTok{(X) }\SpecialCharTok{\%*\%}\NormalTok{ Y}

\CommentTok{\# Define M matrix}
\NormalTok{M }\OtherTok{\textless{}{-}} \FunctionTok{diag}\NormalTok{(}\FunctionTok{ncol}\NormalTok{(X))}

\CommentTok{\# Calculate mu\_beta using the formula from the image}
\NormalTok{mu.beta }\OtherTok{\textless{}{-}} \FunctionTok{solve}\NormalTok{(M }\SpecialCharTok{+} \FunctionTok{t}\NormalTok{(X) }\SpecialCharTok{\%*\%}\NormalTok{ X) }\SpecialCharTok{\%*\%}\NormalTok{ ((}\FunctionTok{t}\NormalTok{(X) }\SpecialCharTok{\%*\%}\NormalTok{ Y) }\SpecialCharTok{+}\NormalTok{ M }\SpecialCharTok{\%*\%}\NormalTok{ beta.hat)}

\CommentTok{\# Calculate A2 using the formula from the image}
\NormalTok{A2 }\OtherTok{\textless{}{-}} \FunctionTok{t}\NormalTok{(Y) }\SpecialCharTok{\%*\%}\NormalTok{ Y }\SpecialCharTok{+} \FunctionTok{t}\NormalTok{(beta.hat) }\SpecialCharTok{\%*\%}\NormalTok{ M }\SpecialCharTok{\%*\%}\NormalTok{ beta.hat }\SpecialCharTok{{-}} \FunctionTok{t}\NormalTok{(mu.beta) }\SpecialCharTok{\%*\%}\NormalTok{ (M }\SpecialCharTok{+} \FunctionTok{t}\NormalTok{(X) }\SpecialCharTok{\%*\%}\NormalTok{ X) }\SpecialCharTok{\%*\%}\NormalTok{ mu.beta}

\CommentTok{\# mu.beta}
\CommentTok{\# A2}
\end{Highlighting}
\end{Shaded}

\begin{Shaded}
\begin{Highlighting}[]
\NormalTok{posterior }\OtherTok{\textless{}{-}} \ControlFlowTok{function}\NormalTok{(X, M, mu.beta, A2, a, b, }\AttributeTok{n\_iter =} \DecValTok{1000}\NormalTok{) \{}
 
  \CommentTok{\# Placeholder for the samples}
\NormalTok{  beta\_samples }\OtherTok{\textless{}{-}} \FunctionTok{matrix}\NormalTok{(}\ConstantTok{NA}\NormalTok{, }\AttributeTok{ncol =} \DecValTok{6}\NormalTok{, }\AttributeTok{nrow =}\NormalTok{ n\_iter)}
\NormalTok{  sigma2\_samples }\OtherTok{\textless{}{-}} \FunctionTok{numeric}\NormalTok{(n\_iter)}
  
  \CommentTok{\# Use the values of mu.beta and A2 computed previously}
  \CommentTok{\# Assuming that mu.beta and A2 are available in the environment}
  \CommentTok{\# If not, compute them using the code provided previously}
  
  \CommentTok{\# Gibbs sampler}
  \ControlFlowTok{for}\NormalTok{ (i }\ControlFlowTok{in} \DecValTok{1}\SpecialCharTok{:}\NormalTok{n\_iter) \{}
    \CommentTok{\# Sample from the inverse{-}gamma distribution for sigma\^{}2}
\NormalTok{    sigma2\_samples[i] }\OtherTok{\textless{}{-}} \DecValTok{1} \SpecialCharTok{/} \FunctionTok{rgamma}\NormalTok{(}\DecValTok{1}\NormalTok{, }\AttributeTok{shape =}\NormalTok{ a }\SpecialCharTok{+}\NormalTok{ n}\SpecialCharTok{/}\DecValTok{2}\NormalTok{, }\AttributeTok{rate =}\NormalTok{ b }\SpecialCharTok{+}\NormalTok{ A2}\SpecialCharTok{/}\DecValTok{2}\NormalTok{)}
    
    \CommentTok{\# Sample from the multivariate normal distribution for beta}
\NormalTok{    beta\_samples[i, ] }\OtherTok{\textless{}{-}} \FunctionTok{mvrnorm}\NormalTok{(}\DecValTok{1}\NormalTok{, mu.beta, sigma2\_samples[i] }\SpecialCharTok{*} \FunctionTok{solve}\NormalTok{(M }\SpecialCharTok{+} \FunctionTok{t}\NormalTok{(X) }\SpecialCharTok{\%*\%}\NormalTok{ X))}
\NormalTok{  \}}

  \FunctionTok{return}\NormalTok{(}\FunctionTok{list}\NormalTok{(}\AttributeTok{beta\_samples =}\NormalTok{ beta\_samples, }\AttributeTok{sigma2\_samples =}\NormalTok{ sigma2\_samples))}
\NormalTok{\}}
\end{Highlighting}
\end{Shaded}

\hypertarget{question-1c}{%
\section{Question 1c}\label{question-1c}}

\hypertarget{question-1ci}{%
\section{Question 1c(i)}\label{question-1ci}}

The trace plots show the value of each regression coefficient over the
iterations of the Gibbs sampler. The trace plot is useful for assessing
the convergence of the Gibbs sampler and the mixing of the Markov chain.

The trace plots show that the Markov chain has converged and the samples
are well-mixed.

\hypertarget{question-1cii}{%
\section{Question 1c(ii)}\label{question-1cii}}

The density plots show the posterior distribution of each regression
coefficient. The density plots provide a visual representation of the
uncertainty in the estimates of the regression coefficients.

Credibility intervals differ from confidence intervals in that they are
based on the posterior distribution of the parameter of interest, rather
than the sampling distribution. The credibility intervals provide a
range of values within which the true parameter value is likely to lie,
given the data and the model.

\begin{longtable}[]{@{}lllllll@{}}
\caption{95\% credibility intervals of the regression
coefficients}\tabularnewline
\toprule\noalign{}
Interval & intercept & B1 & B2 & B3 & B4 & B5 \\
\midrule\noalign{}
\endfirsthead
\toprule\noalign{}
Interval & intercept & B1 & B2 & B3 & B4 & B5 \\
\midrule\noalign{}
\endhead
\bottomrule\noalign{}
\endlastfoot
2.5\% & 0.906 & -0.252 & 9.700 & -0.226 & 1.735 & -3.173 \\
97.5\% & 1.239 & 0.057 & 10.031 & 0.116 & 2.048 & -2.818 \\
\end{longtable}

\hypertarget{question-1ciii}{%
\section{Question 1c(iii)}\label{question-1ciii}}

Using the credibility intervals, the variables worth keeping in the
model are those whose credibility intervals do not include zero. In this
case, the variables X2, X4, X5 are worth keeping in the model, as their
credibility intervals do not include zero.

\hypertarget{question-1d}{%
\section{Question 1d}\label{question-1d}}

We can use \([\sigma^2|\beta,y,X]\) but that would show us the variation
of the in the data that is not explained by the model, this however does
not give us the effect of each predictor variable.
\([\beta|\sigma^2,y,X]\) gives us the effect of each variable, which is
the goal.

\hypertarget{question-2}{%
\section{Question 2}\label{question-2}}

\hypertarget{question-2a}{%
\section{Question 2a}\label{question-2a}}

\hypertarget{question-2b}{%
\section{Question 2b}\label{question-2b}}

The occurence probability for any new search is the probability that the
fisherman is in a particular cell (that is \(Y_i\) = 1 ) given thatwe
could not find him in the previous search (that is \(Z_i\) = 0). The
occurence probability therefore follows the posteria distribution
\(\pi(Y_i = 1|Z_i = 0)\).

\hypertarget{question-2c}{%
\section{Question 2c}\label{question-2c}}

The fisherman was found on the second hour, second search cell. The
posterior distribution for the location of the fisherman increased to
0.038 at the time he was found, from 0.035 at the time of the first
search.

\begin{Shaded}
\begin{Highlighting}[]
\CommentTok{\# Generate the prior distribution for the location of the lost fisherman}
\NormalTok{prior\_distribution }\OtherTok{\textless{}{-}} \FunctionTok{generate\_lost}\NormalTok{(search\_size, }\DecValTok{1000}\NormalTok{)}

\CommentTok{\# Generate the true location of the lost fisherman}
\NormalTok{true\_grid }\OtherTok{\textless{}{-}} \FunctionTok{generate\_fisherman}\NormalTok{(search\_size)}

\NormalTok{initial\_posteria\_distribution }\OtherTok{\textless{}{-}}\NormalTok{ prior\_distribution }\SpecialCharTok{*}\NormalTok{ detect\_pr}
\end{Highlighting}
\end{Shaded}

\begin{Shaded}
\begin{Highlighting}[]
\CommentTok{\# Search and rescue simulation}
\NormalTok{hours\_passed }\OtherTok{\textless{}{-}} \DecValTok{0}
\NormalTok{found }\OtherTok{\textless{}{-}} \ConstantTok{FALSE}

\ControlFlowTok{while}\NormalTok{(}\SpecialCharTok{!}\NormalTok{found }\SpecialCharTok{\&}\NormalTok{ hours\_passed }\SpecialCharTok{\textless{}} \DecValTok{48}\NormalTok{)\{}
\NormalTok{  posteria\_distribution }\OtherTok{\textless{}{-}}\NormalTok{ prior\_distribution }\SpecialCharTok{*}\NormalTok{ detect\_pr}
\NormalTok{  target\_cell }\OtherTok{\textless{}{-}} \FunctionTok{which}\NormalTok{(posteria\_distribution }\SpecialCharTok{==} \FunctionTok{max}\NormalTok{(posteria\_distribution), }\AttributeTok{arr.ind =} \ConstantTok{TRUE}\NormalTok{)}
  
  \CommentTok{\# Simulate search with a Bernoulli trial}
\NormalTok{  found }\OtherTok{\textless{}{-}} \FunctionTok{rbinom}\NormalTok{(}\DecValTok{1}\NormalTok{, }\DecValTok{1}\NormalTok{, detect\_pr[target\_cell[}\DecValTok{1}\NormalTok{], target\_cell[}\DecValTok{2}\NormalTok{]]) }\SpecialCharTok{==} \DecValTok{1}
  \ControlFlowTok{if}\NormalTok{ (found)\{}
    \FunctionTok{print}\NormalTok{(}\FunctionTok{paste}\NormalTok{(}\StringTok{"Fisherman found at"}\NormalTok{, target\_cell[}\DecValTok{1}\NormalTok{], target\_cell[}\DecValTok{2}\NormalTok{], }\StringTok{"in hour"}\NormalTok{, hours\_passed }\SpecialCharTok{+} \DecValTok{1}\NormalTok{))}
\NormalTok{  \} }\ControlFlowTok{else}\NormalTok{ \{}
    \CommentTok{\# Update prior\_distribution using Bayes’ Theorem}
\NormalTok{    prior\_distribution[target\_cell[}\DecValTok{1}\NormalTok{], target\_cell[}\DecValTok{2}\NormalTok{]] }\OtherTok{\textless{}{-}} \DecValTok{0}
\NormalTok{    prior\_distribution }\OtherTok{\textless{}{-}}\NormalTok{ prior\_distribution }\SpecialCharTok{/} \FunctionTok{sum}\NormalTok{(prior\_distribution)}
\NormalTok{    hours\_passed }\OtherTok{\textless{}{-}}\NormalTok{ hours\_passed }\SpecialCharTok{+} \DecValTok{1}
\NormalTok{  \}}
\NormalTok{\}}

\ControlFlowTok{if}\NormalTok{ (}\SpecialCharTok{!}\NormalTok{found)\{}
  \FunctionTok{print}\NormalTok{(}\StringTok{"Fisherman not found in 48 hours."}\NormalTok{)}
\NormalTok{\}}
\end{Highlighting}
\end{Shaded}

\hypertarget{question-2d}{%
\section{Question 2d}\label{question-2d}}

If the detection distribution was a constant over the search grid, then
the combined probability distribution would be the same as the prior
distribution. This is because the detection distribution would not
provide any additional information to the search crew.

This will simplify the search process as the search algorithim would not
have to update the prior distribution using Bayes' Theorem. The
algorithm would simply search the cell with the highest probability of
containing the fisherman at each time step.

\end{document}
